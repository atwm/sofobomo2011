\documentclass[12pt]{article}

\usepackage[dvipsnames]{pstricks}
\usepackage{lipsum}

\usepackage{graphicx}
% where can I find the photos that will be imported for this version
\graphicspath{{./photos.book/}}

% separate out font configuration
%
% Senigallia Cemetery. The Old Part.
% (c) 2011 Mauro Taraborelli (MauroTaraborelliPhoto@gmail.com).
% This work is licensed under a Creative Commons Attribution-Share Alike
% 3.0 Unported License
% See http://creativecommons.org/licenses/by-sa/3.0/
%
% Mauro Taraborelli makes no representation about the suitability or accuracy
% of this software or data for any purpose, and makes no warranties,
% either express or implied, including merchantability and fitness for a
% particular purpose or that the use of this software or data will not
% infringe any third party patents, copyrights, trademarks, or other
% rights.  The software and data are provided "as is". 
% 
% Derived from the work 'Chickens, Anyone?' by Eric R. Jeschke
% http://redskiesatnight.com/books/pod/latex-templates-for-pod-publishing-with-blurb-com/
% (c) 2006 Eric R. Jeschke (eric@redskieatnight.com).
%
% [common fonts definitions]
%
% You might be able to use opentype tools to find out the names of
% the fonts you can use here, or you can just experiment
%  $ otfinfo -a .../*.otf
%
\usepackage{fontspec}% font selecting commands
\usepackage{xunicode}% unicode character macros
\usepackage{xltxtra} % a few fixes and extras

\defaultfontfeatures{Mapping=tex-text,Scale=MatchLowercase}     % to support TeX conventions like ‘‘--
\setmainfont
[ UprightFont = {* Regular} ,
  BoldFont = {* Bold} ,
  ItalicFont = {* Italic} ,
  SmallCapsFont = {Fontin SmallCaps} ] % Can't use * for Fontin SmallCaps. See http://web.archiveorange.com/archive/v/tMRBZ9YChLvsIQr9Lqh2
{Fontin}
\setsansfont
[ UprightFont = {* Regular} ,
  BoldFont = {* Bold} ,
  ItalicFont = {* Italic} ,
  SmallCapsFont = {Fontin Sans Small Caps} ]
{Fontin Sans}
\setmonofont
[ UprightFont = {*} ,
  BoldFont = {* Bold} ,
  ItalicFont = {* Italic} ]
{Lekton}

% separate out color configuration
%
% Senigallia Cemetery. The Old Part.
% (c) 2011 Mauro Taraborelli (MauroTaraborelliPhoto@gmail.com).
% This work is licensed under a Creative Commons Attribution-Share Alike
% 3.0 Unported License
% See http://creativecommons.org/licenses/by-sa/3.0/
%
% Mauro Taraborelli makes no representation about the suitability or accuracy
% of this software or data for any purpose, and makes no warranties,
% either express or implied, including merchantability and fitness for a
% particular purpose or that the use of this software or data will not
% infringe any third party patents, copyrights, trademarks, or other
% rights.  The software and data are provided "as is". 
% 
% Derived from the work 'Chickens, Anyone?' by Eric R. Jeschke
% http://redskiesatnight.com/books/pod/latex-templates-for-pod-publishing-with-blurb-com/
% (c) 2006 Eric R. Jeschke (eric@redskieatnight.com).
%
% [common color definitions]

% Old Gold
\definecolor{MyGold}{rgb/cmyk}{.81,.71,.23/0,.13,.71,.19}

% Medium Gray
\definecolor{MyGray}{rgb/cmyk}{0.38,0.38,0.38/0,0,0,.62}

% Black
\definecolor{MyBlack}{rgb/cmyk}{0,0,0/0.70,0.50,.30,1}

% White
\definecolor{MyWhite}{rgb/cmyk}{1,1,1/0,0,0,0}


% for Blurb, set this to the actual desired size of the cover as
% indicated by the size calculator
\usepackage
[ paperwidth=19.431in ,
  paperheight=8.25in , 
  margin=0in
]
{geometry}

% paragraph indentations
\setlength{\parindent}{0in}
% amount of space before each new paragraph begins
\setlength{\parskip}{0in}

\title{Senigallia Cemetery | Il cimitero di Senigallia}
\author{Mauro Taraborelli}

%%%%%%%%% PDF/X-3 stuff, necessary for Blurb IF USING xelatex %%%%%%%%%
\special{pdf:docinfo <<
/Title (Senigallia Cemetery, The Old Part)   % set your title here
/Author (Mauro Taraborelli)       % set author name
/Subject (Senigallia Cemetery)          % set subject
/Keywords (cemetery, colombarium, gates, flowers, decorations, Italy, photography) % set keywords
/Trapped (False)
/GTS_PDFXVersion (PDF/X-3:2002)
% must have a trim box, but I think Blurb ignores the values
/TrimBox [0.00000 9.00000 684.36000 585.00000] >>
}
\special{pdf:put @catalog <<
/OutputIntents [ <<
/Info (none)
/Type /OutputIntent
/S /GTS_PDFX
/OutputConditionIdentifier (Blurb.com)
/RegistryName (http://www.color.org/)
>> ] >>
}

% for multilingual work you may need polyglossia;
% change the default language and the others to the ones you want
\usepackage{polyglossia}
\setdefaultlanguage[variant=american]{english}
\setotherlanguages{italian}

% for parallel columns
\usepackage{parallel}

% begin the document and suppress page numbers
\begin{document}
\pagestyle{empty}

% create the box with the front cover picture
\newsavebox\FIBox
\sbox\FIBox{\includegraphics[height=8.25in]{sofobomo2011-2-cover}}
% create the box with the rear cover picture
\newsavebox\FRBox
\sbox\FRBox{\includegraphics[height=4in]{sofobomo2011-1}}

% set up the picture environment
\psset{unit=1in}
\begin{pspicture}(\paperwidth,\paperheight)

% create a black background
\psframe[fillstyle=solid,fillcolor=black](0,0)(19.431,8.25)
%\psframe[fillstyle=solid,fillcolor=white](9.625,0)(9.806,8.25)

% place the front cover picture
\rput[lb](9.806,0){\usebox\FIBox}
% place the rear cover picture
\rput[lb](2.312,0){\usebox\FRBox}

% put the text on the front cover
\rput[lb](10.50,6.35){\fontsize{.4in}{.4in}\selectfont \textbf{\textsf{\textsc{\color{White}{Mauro Taraborelli}}}}}
\rput[lb](10.50,5.45){\fontsize{.6in}{.6in}\selectfont \textbf{\textsf{\color{MyGold}{Senigallia Cemetery}}}}
\rput[lb](10.50,5.10){\fontsize{.6in}{.6in}\selectfont \textbf{\textsf{\color{MyGold}{The Old Part}}}}

% put the text on the spine (note the rotation over 270 degrees)
\rput[l]{270}(9.735,7.75){\textbf{\textsf{\textsc{\color{White}{\scriptsize Mauro Taraborelli}}}}}
\rput[l]{270}(9.735,6.75){\textbf{\textsf{\color{MyGold}{\scriptsize Senigallia Cemetery. The Old Part.}}}}

% put the publisher’s logo on the spine
%\rput[b](9.745,0.75){\color{white}{\fbox{\Logo M}}}

% Create a Box containing the text for the back cover
\newsavebox\Blurbbox
\sbox\Blurbbox{\begin{minipage}{7in}
\textcolor{white}{
\begin{Parallel}{0.47\textwidth}{0.47\textwidth}
\ParallelLText{
    \begin{english}[variant=american]
        \textbf{\textsf{\color{MyGold}{\large Senigallia Cemetery. The Old Part.}}}
    \end{english}
}
\ParallelRText{
    \begin{italian}
        \textbf{\textsf{\color{MyGold}{\large Il cimitero di Senigallia. La parte antica.}}}
    \end{italian}
}
\ParallelPar
\ParallelLText{
    \begin{english}[variant=american]
        Senigallia cemetery is a monumental cemetery that dates back
        to the second half of the 19\textsuperscript{th} century.
        It covers the slopes of one of the hills that surround the town
        and walking trough its rows is of visual and historic interest.
    \end{english}
}
\ParallelRText{
    \begin{italian}
        Il cimitero di Senigallia è un cimitero monumentale che risale
        alla seconda metà del XIX secolo.
        Occupa i pendii di una delle colline che circondano la città
        e camminare tra i suoi sentieri è di interesse visuale e storico.
    \end{italian}
}
\ParallelPar
\ParallelLText{
    \begin{english}[variant=american]
        This book is a photographic journey in the old part of the Senigallia cemetery,
        presented through four visual themes:
        colombarium, gates, flowers and plants, decorations.
    \end{english}
}
\ParallelRText{
    \begin{italian}
        Questo libro è un viaggio fotografico nella parte antica del cimitero di Senigallia,
        presentato attraverso quattro temi visuali:
        colombari, cancelli, fiori e piante, decorazioni.
    \end{italian}
}
\ParallelPar
\end{Parallel}
}
\end{minipage}}
% And position the box
\rput[tl](1.312,7.5){\usebox\Blurbbox}

% Then we close all open environments
\end{pspicture}
\end{document}
